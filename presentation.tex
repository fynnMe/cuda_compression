\documentclass{beamer}
\usepackage{listings}
\usepackage{url}

\title{Cuda Kompression - Task 1}
\author{Fynn Meffert \and Theo Reichert}
\date{\today}
\begin{document}

\frame{\titlepage}

\begin{frame}
    \frametitle{Was haben wir gemacht}
    \begin{itemize}
        \item generieren 1 Gigabyte Daten
        \item kopieren Daten auf GPU, berechnen, kopieren zum Host
        \item berechne Daten auf Host
        \item Ergebnisse von Host und GPU sind gleich
     \end{itemize}
\end{frame}

\begin{frame}
    \frametitle{Was haben wir gemacht}
    \begin{itemize}
        \item verschieden Cuda Configs
              berechne dreimal mit einer Cuda Config jeweils unterschiedlichen Daten
              um Durchschnitt zu bilden
         \begin{figure}
             \centering
             \includegraphics[width=0.8\linewidth]{cuda_configs.jpg} % Replace with your image file
         \end{figure}
    \end{itemize}
\end{frame}

\begin{frame}
    \frametitle{Was haben wir gemacht}
    \begin{itemize}
        \item beste Cuda Config bei variierender Arraygröße
          \begin{figure}
              \centering
              \includegraphics[width=0.8\linewidth]{cuda_runtime_vs_array_size.jpg}
          \end{figure}
    \end{itemize}
\end{frame}

\begin{frame}
    \frametitle{Fragen}
    Nvidia RTX 8000
    \begin{itemize}
        \item 4608 CUDA cores
        \item 72 SMs mit je 64 Threads
        \item Turing Architektur → maximal 16 Thread-Blocks pro SM.
              Ist $72 \cdot 16 = 1152$ maximale Anzahl rechnender Blöcke
              auf ganzer GPU zu einem Zeitpunkt?
    \end{itemize}
\end{frame}

\begin{frame}
    \frametitle{Fragen}
    \begin{itemize}
        \item Was ist in der Aufgabenstellung mit "static bit size" (Seite 5) gemeint?
    \end{itemize}
\end{frame}

\begin{frame}[fragile]
    \frametitle{Fragen}
    \framesubtitle{Kompression}
    Daten auf dem Bus zwischen Host und GPU, wenn bit-gepackt in uin16
    \begin{lstlisting}
unkomprimiertes Beispiel
a: 0000 0xxx
b: 000x xxxx
c: 0xxx xxxx
d: 00xx xxxx

komprimiertes Bsp.   gepackte bit Folge
a:       xxx                                aaa
b:    x xxxx                          bbbb baaa
c:  xxx xxxx                 ccc cccc bbbb baaa
d:   xx xxxx         d dddd dccc cccc bbbb baaa
    \end{lstlisting}
    a, b und c passen in einen uint16, aber d nicht noch zusätzlich.
\end{frame}

\begin{frame}[fragile]
    \frametitle{Fragen}
    \framesubtitle{Kompression}
    Frage: Wäre uint16 compressed bit-string dann
           "0ccc cccc bbbb baaa"?

    Da jeder komprimierte String vom 1 startet müssen Nullen als
    Präfix Lückenfüller sein.\\~

    Frage: Wie dekomprimiert man?\\~
    Antwort: Gar nicht. Man fügt Null zwischen zwei komprimierten Zahlen ein um
                        Overflow zu verhindern und kann so ganz normal rechen,
                        wobei es problematisch wird, wenn die gepackten Summanden
                        unterschiedlich fragmentiert sind:
    \begin{lstlisting}
            "0ccc ccc0 bbbb 0aaa"
          + "0hh0 gfff f0ee 0ddd"
    \end{lstlisting}
\end{frame}

\begin{frame}[fragile]
    \frametitle{Fragen}
    \framesubtitle{Kompression}
    Ist das die erwartete Interpretation?
    \begin{lstlisting}
        "000 00 0cccccc 0bbbb 0aaa"
      + "0hh 0g 000ffff 000ee 0ddd"
    \end{lstlisting}
\end{frame}

%\begin{frame}
%    \frametitle{Probleme}
%\end{frame}

\end{document}
