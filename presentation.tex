\documentclass{beamer}
\usepackage{listings}
\usepackage{url}

\title{GPU/CUDA meets Database Systems}
\author{Fynn Meffert \and Theo Reichert}
\date{\today}
\begin{document}

\frame{\titlepage}

\begin{frame}
    \frametitle{Problem}
    query processing on GPU is heavily Memory bound
    % nvidia profiler screenshot
\end{frame}

\begin{frame}
    \frametitle{Task}
    \begin{itemize}
        \item compress data in global memory with "zero supression"
          (removes leading zeros)
        \item more logical data in one e.g. `uint\_64`
        \item incorporate `shared memory` and `atomic` operations
        \item measure throughput of program with different cuda configs
          and improve approach
     \end{itemize}
\end{frame}

\begin{frame}[fragile]
    \frametitle{Algorithmic sketch}
    \framesubtitle{compression - "zero supression"}
    \begin{lstlisting}
compressed data
  0bbbbbbb 0aaaaaaa

decompressed data
   00000000 0aaaaaaa
   00000000 0bbbbbbb
    \end{lstlisting}
\end{frame}


\begin{frame}
    \frametitle{Algorithmic sketch}
    \framesubtitle{uncompressed implementation}
         \begin{figure}
             \centering
             \includegraphics[width=1\linewidth]{uncompressed_kernel_code.png} % Replace with your image file
         \end{figure}
\end{frame}


\begin{frame}
    \frametitle{Algorithmic sketch}
    \framesubtitle{naive compressed implementation}
    \begin{itemize}
        \item load array of `uint\_64` from global memory
        \item decompress logical data from `uint\_64`
        \item do given operation
        \item compress logical data into `uint\_64`s
        \item write array of `uint\_64` to global memory
     \end{itemize}
\end{frame}

\begin{frame}
    \frametitle{Measurement}
    \framesubtitle{different cuda configs}
       \begin{figure}
           \centering
           \includegraphics[width=1\linewidth]{uncompressed_configs.png} % Replace with your image file
       \end{figure}
\end{frame}

\begin{frame}
    \frametitle{Measurement}
    \framesubtitle{different cuda configs}
       \begin{figure}
           \centering
           \includegraphics[width=1\linewidth]{compressed_configs.png} % Replace with your image file
       \end{figure}
\end{frame}

\begin{frame}
    \frametitle{Measurement}
    \begin{itemize}
        \item uncompressed
         \begin{figure}
             \centering
             \includegraphics[width=0.4\linewidth]{uncompressed_configs.png} % Replace with your image file
             \includegraphics[width=0.5\linewidth]{uncompressed_1GiB_1024_2048.profile.png} % Replace with your image file
         \end{figure}

         \item compressed
         \begin{figure}
             \centering
             \includegraphics[width=0.4\linewidth]{compressed_configs.png} % Replace with your image file
             \includegraphics[width=0.5\linewidth]{compressed_1GiB_1024_2048.profile.png} % Replace with your image file
         \end{figure}
    \end{itemize}
\end{frame}

\begin{frame}
    \frametitle{Measurement}
    \framesubtitle{varying array sizes}
    \begin{figure}
        \centering
        \includegraphics[width=0.9\linewidth]{uncompressed_array_sizes.png} % Replace with your image file
    \end{figure}
\end{frame}

\begin{frame}
    \frametitle{Measurement}
    \framesubtitle{varying array sizes}
     \begin{figure}
         \centering
         \includegraphics[width=0.9\linewidth]{compressed_array_sizes.png} % Replace with your image file
     \end{figure}
\end{frame}

\begin{frame}
    \frametitle{Measurement}
    \framesubtitle{speedup vs bitsize}
    \begin{figure}
        \centering
        \includegraphics[width=0.9\linewidth]{speedup_vs_bitsize.png} % Replace with your image file
    \end{figure}
\end{frame}

\begin{frame}
    \frametitle{Measurement}
    \framesubtitle{speedup vs bitsize}
    \begin{figure}
        \centering
        \includegraphics[width=0.9\linewidth]{speedup_vs_bitsize_zoomed.png} % Replace with your image file
    \end{figure}
\end{frame}

\begin{frame}
    \frametitle{Learnings}
    \begin{itemize}
        \item sometimes the easy approach is a good approach
        \item synchronization overhead is no joke
        \item basic workflow in CUDA programming and debugging
        \item usage of shared memory
    \end{itemize}
\end{frame}



%\begin{frame}[fragile]
%    \frametitle{Algorithmic scetch}
%    \framesubtitle{compression}
%    \begin{lstlisting}
%uncompressed example
%a: 0000 0xxx
%b: 000x xxxx
%c: 0xxx xxxx
%d: 00xx xxxx
%
%compressed example        packed bit string
%a:       xxx                                aaa
%b:    x xxxx                          bbbb baaa
%c:  xxx xxxx                 ccc cccc bbbb baaa
%d:   xx xxxx         d dddd dccc cccc bbbb baaa
%    \end{lstlisting}
%    a, b and c fit into the `uint16`, but d does not additionally fit
%\end{frame}
%
%\begin{frame}[fragile]
%    \frametitle{Fragen}
%    \framesubtitle{compression}
%    Frage: Wäre uint16 compressed bit-string dann
%           "0ccc cccc bbbb baaa"?
%
%    Da jeder komprimierte String vom 1 startet müssen Nullen als
%    Präfix Lückenfüller sein.\\~
%
%    Frage: Wie dekomprimiert man?\\~
%    Antwort: Gar nicht. Man fügt Null zwischen zwei komprimierten Zahlen ein um
%                        Overflow zu verhindern und kann so ganz normal rechen,
%                        wobei es problematisch wird, wenn die gepackten Summanden
%                        unterschiedlich fragmentiert sind:
%    \begin{lstlisting}
%            "0ccc ccc0 bbbb 0aaa"
%          + "0hh0 gfff f0ee 0ddd"
%    \end{lstlisting}
%\end{frame}
%
%\begin{frame}[fragile]
%    \frametitle{Fragen}
%    \framesubtitle{Kompression}
%    Ist das die erwartete Interpretation?
%    \begin{lstlisting}
%        "000 00 0cccccc 0bbbb 0aaa"
%      + "0hh 0g 000ffff 000ee 0ddd"
%    \end{lstlisting}
%\end{frame}

%\begin{frame}
%    \frametitle{Probleme}
%\end{frame}

\end{document}
